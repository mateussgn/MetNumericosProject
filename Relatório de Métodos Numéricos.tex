\documentclass[a4paper, 12pt]{article}
\usepackage[top=2cm, right=3cm, bottom=3cm, left=2cm]{geometry}
\usepackage[utf8]{inputenc}
\usepackage{amsmath, amsfonts, amssymb}

\begin{document}
	\begin{center}
		\textbf{Projeto de Métodos Numéricos 2019.1}
	\end{center}
%começo do sumário
	\begin{center}
		\textbf{Sumário}
	\end{center}
	\begin{flushleft}		
		\textbf{I. Introdução}
		\item
		\textbf{II. Problema 1}
		\begin{enumerate}
			\item
				Problema
			\item
				Modelo
			\item
				Solução Analítica
			\item
				Gráficos das Soluções Numéricas
			\item
				O que acontece quando t tende ao ifinito?
		\end{enumerate}
		\textbf{III. Problema 2}
		\begin{enumerate}
			\item
				Problema
			\item
				Modelo
			\item
				Solução Analítica
			\item
				Gráficos das Soluções Numéricas
			\item
				O que acontece quando t tende ao ifinito?
		\end{enumerate}
		\textbf{III. Problema 3}
		\begin{enumerate}
			\item
				Problema
			\item
				Modelo
			\item
				Solução Analítica
			\item
				Gráficos das Soluções Numéricas
			\item
				O que acontece quando t tende ao ifinito?
		\end{enumerate}
		\textbf{IV. Problema 4}
		\begin{enumerate}
			\item
				Problema
			\item
				Modelo
			\item
				Solução Analítica
			\item
				Gráficos das Soluções Numéricas
			\item
				O que acontece quando t tende ao ifinito?
		\end{enumerate}
	\end{flushleft}
%fim do sumário
%começo dos capítulos
	\begin{flushleft}		
		\textbf{I. Introdução}
		\item
		\textbf{II. Problema 1}
		\begin{enumerate}
			\item
				Problema
				\begin{enumerate}
					\item
						Temos que:
						$$\dfrac{dP}{dt} = -kP(t)$$
						Multiplicando ambos os lados por \underline{dt} e isolando o \underline{-k} do \underline{P(t)} temos:
						$$\dfrac{dP}{P(t)} = -kdt \;\;(2)$$
						Usando o método das equações separáveis temos:
						$$\int\dfrac{dP}{P(t)} = -\int kdt$$
						Resolvendo a integral temos:
						$$ln[P(t)] = -kt+C$$
						$$\textit{e}^{ln[P(t)]} = \textit{e}^{-kt+C}$$
						$$P(t) = \textit{e}^{-kt+C}$$
						$$P(t) = \textit{e}^{-kt}\cdot\textit{e}^C$$
						$$P(t) = C\textit{e}^{-kt}$$
						Para achar a constante \underline{k}, usamos o dado fornecido no enunciado:
						$$3P(t) = P(t+10)$$
						$$3C\textit{e}^{-kt} = C\textit{e}^{-k(t+10)}$$
						$$3C\textit{e}^{-kt} = C\textit{e}^{-kt-10k}$$
						$$3C\textit{e}^{-kt} = C\textit{e}^{-kt}\cdot\textit{e}^{-10k}$$
						$$3 = \textit{e}^{-10k}$$
						$$ln (3) = ln(\textit{e}^{-10k})$$
						$$ln (3) = -10k\cdot ln(\textit{e})$$
						$$-10k = ln(3)$$
						$$k = -\dfrac{ln (3)}{10}$$
						Resolvendo:
						$$P(20) = P(0)e^{\frac{ln (3)}{10}\cdot 20}$$
						$$150000 = P(0)e^{2ln(3)}$$
						$$P(0) = 16.666,667$$
					\item
				\end{enumerate}
			\item
				Gráficos das Soluções Numéricas
			\item
				O que acontece quando t tende ao ifinito?
		\end{enumerate}
		\textbf{III. Problema 2}
		\begin{enumerate}
			\item
				Problema
				\begin{enumerate}
					\item
						$\dfrac{dA}{dt} =\;$taxa de entrada - taxa de saída. Usando os valores fornecidos no enunciado:					
						$$\dfrac{dA}{dt} = 2r - r\dfrac{A}{300}$$
						Onde \underline{r} é o fluxo de entrada dos galões.
						
						Usando o método de fator integrante:
						$$\dfrac{dA}{dt}+ r\dfrac{A}{300} = 2r$$
						$$\dfrac{dA}{dt}\cdot\mu+ r\dfrac{A}{300}\cdot\mu = 2r\cdot\mu$$
						Onde $\mu$ = $\textit{e}^{\int \frac{r}{300}dt}$.
						$$\dfrac{d}{dt}(A\textit{e}^{\frac{rt}{300}}) = 2r\textit{e}^{\frac{rt}{300}}$$
						Integrando ambos os lados:
						$$A\textit{e}^{\frac{rt}{300}} = 600\dfrac{r\textit{e}^{\frac{rt}{300}}}{r}+C$$
						$$A = 600 + \dfrac{C}{\textit{e}^{\frac{rt}{300}}}$$
						$$A(t) = 600 + C\textit{e}^{-\frac{rt}{300}}$$
						Achando valor de \underline{C}:
						$$50 = 600 + C$$
						$$C = -550$$
						Substituindo em \underline{A(t)}:
						$$A(t) = 600 - 550\textit{e}^{-\frac{rt}{300}}$$
					\item
						$$A(100) = 600 - 550\textit{e}^{-1}$$
						$$A(100) = 397,666\;ll$$
					\item
				\end{enumerate}
			\item
				Gráficos das Soluções Numéricas
			\item
				O que acontece quando t tende ao ifinito?
		\end{enumerate}
		\textbf{III. Problema 3}
		\begin{enumerate}
			\item
				Problema
				\begin{enumerate}
					\item
						Usando T = 0:
						$$T = C\cdot\textit{e}^{kt} + T_n$$
						$$T = C + T_n$$
						$$C = T - T_n$$
						Usando os dados do enunciado:
						$$T = (T_0 - T_n)\textit{e}^{kt} + T_n$$
						$$T - T_n= (T_0 - T_n)\textit{e}^{3k}$$
						$$200 - 70 = (300 - 70)\textit{e}^{3k}$$
						$$130 = 230\textit{e}^{3k}$$
						$$\textit{e}^{3k} = \dfrac{130}{230}$$
						$$ln(\textit{e}^{3k}) = ln(\dfrac{13}{23})$$
						$$3k\cdot ln(\textit{e}) = ln(\dfrac{13}{23})$$
						$$k = \dfrac{ln(\dfrac{13}{23})}{3}$$
						$$k = -\dfrac{2}{10}$$
						
						$$75 - 70 = 230\textit{e}^{-\frac{2t}{10}}$$
						$$\textit{e}^{-\frac{2t}{10}} = \dfrac{5}{230}$$
						$$\textit{e}^{-\frac{2t}{10}} = \dfrac{1}{46}$$
						$$ln(\textit{e}^{-\frac{2t}{10}}) = ln(\dfrac{1}{46})$$
						$$-\dfrac{2t}{10}\cdot ln(\textit{e}) = ln(\dfrac{1}{46})$$
						$$t = -\dfrac{ln(\dfrac{1}{46})}{5}$$
						$$t = 19,14 \;min$$
					\item
				\end{enumerate}
			\item
				Gráficos das Soluções Numéricas
			\item
				O que acontece quando t tende ao ifinito?
		\end{enumerate}
	\end{flushleft}
%fim dos Capítulos
\end{document}